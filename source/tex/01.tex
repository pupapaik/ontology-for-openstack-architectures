
\section{Introduction}

%proč OpenStack - protože komunita 500000 vývojářů, tisíce firem, nárůst kódu za poslední dobu stacalytics.com 
% How OpenStack Works?

%vendor lockin, scalability from notebook to thousans of servers like CERN

% 2,292 Companies; 8,066 Individual Members; 130 Countries; 2,007 Total Contributors; 341 Average Monthly Contributors; 89,156 Code Contributors - more information at 
% www.stackalytics.com July 2014

OpenStack is the largest open-source cloud computing platform today. Many companies participate to its code, extend core functions and write new service backends to fit their business goals. The actual system consists of many components designed with plugin architecture that allows custom implementations for various service backends. These components can be combined and configured to match available software and hardware resources and real use-case needs.

Each implementation has its own component combination and use some form of configuration management tool to enforce the service states on designated servers and possibly other network components. These tools require data that covers configuration of all components. Detecting component inconsistencies by hand is painful and time consuming process.

We propose a formalization of OpenStack service architecture model, based on the approaches developed in classic knowledge representation domain, especially Service-Oriented Architecture by OpenGroup. Component definition is encoded in an ontology using the standard OWL-DL language, which enables sharing of knowledge about configurations across various systems. Reasoning can be used on the specification to automate validation of configuration changes.

When dealing with hundreds of components with thousands of properties and relations, keeping track of changes throughout its life cycle is very challenging. Current approaches are ad hoc, even OpenStack Fuel has severe limitations, there exists no standard for specifying common OpenStack architectural model. The question how to convert the proposed OWL-DL schema to metadata format that configuration management tools can process is discussed. We are working on external node classification service that uses graph database to serialize the OWL ontology with REST API that configuration management tools can use as metadata provider. This can streamline the process of adopting new services and service backends in predictable manner.

\subsection{Use Cases}

OpenStack is system that has growing number of components with growing number of components and drivers. As we will show in following text there's no universal installation of OpenStack.  

% představit OpenStack jako systém s rolemi, konfiguracemi, komponenty, drivery. jina instalace per use case. Nexexistuje univerzální instalace. Vysoka komplexita, services

\subsection{Infrastructure Modeling}

Tools usually collect required metadata through web forms or answer files. This is not a conceptual way to describe model.

% Jak vytvořit high level model (logické schéma) architektury? a přenést ji do low level design realizace? 
% Jak správně definovat architekturu na základě hw infrastruktury a target use case?

\subsection{Process Automation}

% Jak celý proces deploymentu automatizovat?

